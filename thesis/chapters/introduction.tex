Text is ubiquitous and is arguably one of the most important information mediums  to this date.
Every day and in numerous domains, new texts are produced.
Due to the number of digitally available text, processing texts automatically instead of manually becomes more appealing and in some cases even necessary.
However, language itself is an inherently ambiguous information medium, which in turn significantly hardens the task of processing it automatically.
Still, numerous approaches have been devised to extract information from text to solve different natural language processing tasks, such as text classification/categorization \cite[p.~575]{Manning2000}, summarization \cite{Mani1999} or language translation \cite{Weaver1955}, to name but a few.

In this work, we concentrate on text classification \cite[p.~232]{Manning2000}, or categorization, which entails the task of predicting the class of a text document based on its content.
In this context, finding an appropriate representation of text is crucial and can greatly influences the success of solving the task.
More conventional text representations are often based on counting the words of a text and representing the text content by these word counts.
The \textit{Bag-Of-Word} \cite[p.~237]{Manning2000}, or BoW, representation is a widely used example of such a procedure which has also shown great real-world performance on several tasks, not only in text classification.
However, simple word-count based text representations come at the expense of losing information about their underlying text, eg. the sentence structure and word order.
That said, there are several other, more sophisticated text representations, for instance co-occurrence graphs \cite{Rousseau2013}, concept maps \cite{Novak2008,Falke2017b} or document embeddings \cite{Dai2015,Lau2016}.
Each of these representations is also subject to different trade-offs, for instance in the ability to capture the needed information and its compute time.

In our work, we are interested in graph-based text representations, especially concept maps \cite{Novak2008,Falke2017b}, to overcome the aforementioned shortcomings of simple text representations like \textit{BoW}.
%We explore the usefulness of more structured graph representations in the context of text classification.
For this, we first generate concept maps for several text corpora and subsequently devise a number of experiments to gain a better insight into the retained information and structure of these concept maps.
Additionally, we capitalize on so-called graph kernels \cite{Kulharia2008} to operate on the concept maps which in turn enables us to perform graph classification.
At the core of this works stands the question how we can use graph-based approaches to improve on current, conventional text representations like BoW.
Similar ways to use graph-based text representations in the context of text classification have been explored in the literature, namely using co-occurrence graphs \cite{Rousseau2015a, Nikolentzos2017a}, DRS graphs \cite{Gaspar2011} and concept graphs \cite{Gulrandhe2015} also.
In Section \ref{subsec:graph_kernel_based_text_classification} we will both introduce the history of this approach more throughly and explain the differences to our work.

\paragraph{Text Classification}
The need for automatic text classification appears in many fields as the number of digitally available texts grows daily and makes manually classifying infeasible if not impossible.
There are numerous real-world applications for text classification, ranging from automatically determining whether a given email is spam or not \cite{Yu2008}, to sentiment analysis \cite{Liu2012}.
Finding ways to train an automatic classifier which can reliably and accurately classify raw texts therefor becomes appealing.
There is a rich history of research on text classification and several approaches have been devised to automatically process natural language texts.
For an overview and introduction into natural language processing, see \cite{Manning2000}.

\paragraph{Graph Classification}
Trees, sequences, networks and other graphs occur in a lot of contexts.
Because of their structured nature, graphs are perfect candidates to capture connected data, or datapoints with relations between them.
Yet, operating on graphs is often challenging, precisely because of its structured and often non-linear traits, since graphs are often of non-fixed size and their structure can vary greatly.
Nevertheless, finding ways to automatically process graphs becomes more important as they naturally turn up in many contexts, for instance in social networks or transaction histories to name but a few.
Especially the task of automatic graph classification has interesting applications, ranging from determining the toxicity of molecules to predicting friends for users in a social network.
In Table \ref{table:graph_classification_examples} we gathered some example applications and previous work in graph classification.

\begin{table}[htb!]
\centering
\renewcommand*{\arraystretch}{0.95}
\begin{tabular}{llllr}
\toprule
Context & Vertices & Edges & Classes &  \\
\midrule
Chemistry & Atoms & Bonds & Toxicity (binary) & \cite{Mahe2005} \\
Biology & Amino Acids & Spatial Links & Protein Types & \cite{Vazquez2008} \\ 
Social Networks & Users & Are Friends & Bot Detection (binary) & \cite{Wang2014} \\
\bottomrule
\end{tabular}%
\caption[Table: Graph Classification Applications]{Graph classification applications.}%
\label{table:graph_classification_examples}
\end{table}

For our work, we explore the usefulness of graph representations of text, namely co-occurrence graphs and concept maps.
In Section \ref{subsec:graphs} we will introduce the graphs and approaches to classify them more throughly.

\labelsection{Hypothesis and Goals}{subsec:hypothesis_and_goals}
In this work, we evaluate the usefulness of graph representations generated from text in the context of text classification. In particular, we work out how or whether we can harness the structure of concept maps to improve text-based classification performance.
The main hypothesis of this work is:
\begin{quote}
\hypothesis
\end{quote}

There are a great number of approaches for text classification, a lot of them based on counting the words of a text and learning a classifier with these frequencies.
Yet, these word-count based approaches often do not take the semantic or syntax of sentences into account, ie. text-based approaches often do not leverage the structure and meaning of sentences.
There are partial solutions for this issue, for instance by augmenting single-word counts with n-grams \cite[p.~191]{Manning2000} counts.
N-grams are sequences of length $n$ of consecutive words in the text, therefore they can, in principle, capture word dependency and word order.
However, their usefulness is limited as they also incur a significant cost by greatly increasing the dimensionality of the resulting BoW feature vectors.
That said, count-based text representations are widely used and achieve high performance both in compute time and classification scores.

In our work, as a possible solution to the aforementioned issues in conventional count-based representations, we explore how concept maps, and other graph representations like co-occurrence graphs, could improve upon or augment existing text classification approaches.
We choose concept maps as our preferred graph representation since they are specially created to capture important concepts and their relation to each other, therefore might also capturing the semantic and other structure of their underlying text.
In the next sections, we will further explain graph representations for text and why they could prove to be a viable addition to the toolbox in text classification.

\labelsection{Thesis Structure}{subsec:thesis_structure}
In the next section, \ref{sec:background} \fullref{sec:background}, we will introduce the concepts used in the rest of this work.
We also offer an overview over related work and the history of the field of text- and graph based classification.

In Section \ref{sec:evaluation} \fullref{sec:evaluation}, we further describe our hypothesis and outline questions regarding it. This section also covers the methodology and experiments we use to answer these questions.

Next, in Section \ref{sec:results} \fullref{sec:results}, we then provide the results to the questions posed in the preceding chapter, interpreting them in the context of our hypothesis.
Here, we also provide related observations regarding our approach.

Finally, in Section \ref{sec:conclusions} \fullref{sec:conclusions}, we gather the results of previous sections into a more high-level picture.
We close with finishing remarks concerning possible further work and also interpret our results in the context of previous work.