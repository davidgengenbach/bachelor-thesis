In Figure \ref{fig:statistics_percentage_multi_word_labels} we see, that most concept map node labels consist of more than one word.

\begin{figure}[htb!]
    \centering
    \includegraphics[width=0.7\linewidth]{assets/figures/tmp/statistics_percentage_multi_word_labels.pdf}
    \caption[Statistics: Percentage multi-word node labels]{Percentage of multi-word to single-word node labels per dataset.}\label{fig:statistics_percentage_multi_word_labels}
\end{figure}

However, with the normal approach, like counting the node labels, these multi-word labels are treated as single labels, effectively discarding important information.
In the experiments of the last question, Question \ref{question:importance_structure}, we actually implicitly used the multi-word labels since we un-rolled the graph into text again and then used conventional text vectorizer, ie. BoW.
In these experiments, we saw an improvement over the graph based approach, possibly due to the multi-word labels being split.
So, for our next graph-based approach, we also split nodes with multi-word labels into multiple single-word nodes.
This is done by splitting the node labels into individual words, than creating nodes with these single words and adding all the (directed) edges of the original node.
When a resulting single-word label is a stopword, the corresponding node is removed from the graph.
See Figure \ref{fig:example_split_labels} for an example.

Additionally, we also evaluate the stemming or lemmatizing [p.~4]\cite{Manning2000} the split single-words.

\begin{figure}[htb!]
    \centering
    \begin{subfigure}[t]{.4\linewidth}  {\includegraphics[width=\linewidth]{assets/figures/tmp/graph_example_split_labels_before.pdf}}
        \caption{Before}
    \end{subfigure}
\hspace{2cm}
    \begin{subfigure}[t]{.4\linewidth}  {\includegraphics[width=\linewidth]{assets/figures/tmp/graph_example_split_labels_after.pdf}}
        \caption{After}
    \end{subfigure}
    \caption[Example: Multi-word node labels Splitting]{Example for multi-word label splits. Here, both the ``concept maps" and ``multi-word concepts" node labels are  split.}\label{fig:example_split_labels}
\end{figure}

In Figure \ref{table:results_multi_label_split} we report the results when doing graph classification with and without multi-word label splitting.
For this experiment, we use the Weisfeiler-Lehman graph kernel to extract the feature maps.
As we can see, the splitting actually improves the performance quite a lot, consistently outperforming the non-split version on all datasets.
On some datasets, the additional stemming of single-word labels after splitting also further improves the score.

\begin{table}[htb!]
    \centering
\begin{tabular}{lrrrr}
\toprule
    {} & \multicolumn{3}{c}{F1 macro}  &   \\
     &  Un-Split  &  Un-Stemmed &  Stemmed &   $p$-Value  \\
    \midrule
    ling-spam       & 0.8160 & 0.8835 & \textbf{0.9131} & 0.0008 \\
    ng20            & 0.4188 & 0.5680 & \textbf{0.5741} & 0.0001 \\
    nyt\_200         & 0.7436 & \textbf{0.8464} & 0.8294 & 0.0001 \\
    r8              & 0.6772 & \textbf{0.8599} & 0.8441 & 0.0001 \\
    review\_polarity & 0.6094 & \textbf{0.7150} & 0.6699 & 0.0036 \\
    rotten\_imdb     & 0.6346 & \textbf{0.7805} & 0.7775 & 0.0001 \\
    ted\_talks       & 0.2436 & 0.3934 & \textbf{0.4126} & 0.0035 \\
    \bottomrule
\end{tabular}
\caption[Results: Multi-label split]{Results for multi-label node label split. The $p$-value is calculated with the best split-word approach, ie. stemmed and un-stemmed, versus the plain version.}\label{table:results_multi_label_split}
\end{table}

The classification scores for this multi-word splitting nearly performs as well as the linearized, BoW approach from the previous Question \ref{question:importance_structure}, ranging from 1\% to 3\% difference in F1 macro.
However, it has to be noted that splitting the multi-word node labels, and subsequently extracting features with WL, produces higher dimensional feature vectors than the un-split approach.


\answersummary{
Splitting the multi-word labels shows the greatest performance improvement for WL, ranging from a raise of 10\% to 24\% better F1 macro score over the un-split approach.
Splitting the node labels and then using WL performs nearly as good as the best graph kernel we saw, the linearized content-only approach from Question \ref{question:importance_structure}.
}