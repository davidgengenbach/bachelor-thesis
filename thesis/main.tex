\documentclass[report,type=bsc,colorback,accentcolor=tud9c,bigchapter,bibliography=totoc,11pt]{tudthesis}
\usepackage[english]{babel}
\usepackage{amsmath}
\usepackage{hyperref}
\usepackage{booktabs}
\usepackage{tabularx}
\usepackage{multirow}
%\usepackage{subfig}
\usepackage{subcaption}
\usepackage{enumitem}
\usepackage[backgroundcolor=lightgray,bordercolor=white]{todonotes}
\usepackage{cellspace}
\usepackage{mdframed}
\usepackage{titlesec}
\usepackage{csquotes}
\usepackage[final]{showlabels}
\usepackage{epigraph}
% \epigraphsize{\small}% Default
\setlength\epigraphwidth{12cm}
\setlength\epigraphrule{0pt}

\usepackage{etoolbox}

%\titlespacing*{\chapter}{0ex}{-1cm}{1cm}

\definecolor{davysgrey}{rgb}{0.33, 0.33, 0.33}
\mdfsetup{%
	linecolor=davysgrey,
	linewidth=1pt,
	%backgroundcolor=red!10,
	roundcorner=4pt
}

\setenumerate{topsep=8pt,parsep=1pt}
\setitemize{topsep=8pt,parsep=1pt}

\setlength\cellspacetoplimit{4pt}
\setlength\cellspacebottomlimit{4pt}

\thesistitle{Text classification using concept maps}{}
\author{David Marlon Gengenbach}
\birthplace{Gemmingen-Stebbach}
\referee{Prof. Dr. Iryna Gurevych}{Prof. Dr. Kristian Kersting}%[Dr. Tobias Falke]
\department{Fachbereich Informatik}
\group{UKP Lab}
%\dateofexam{\today}{\today}
\newcommand{\red}[1]{
    \textcolor{red}{#1}
}

\newcommand{\getmydate}{%
  \ifcase\month%
    \or Januar\or Februar\or M\"arz%
    \or April\or Mai\or Juni\or Juli%
    \or August\or September\or Oktober%
    \or November\or Dezember%
  \fi\ \number\year%
}

\newcommand{\labelsection}[2]{%
    \section{#1}%
    \label{#2}%
}

\newcommand{\labelsectioninclude}[3]{%
    \labelsection{#1}{#2}%
    \input{#3}%
}

\newcommand{\labelsubsection}[2]{%
    \subsection{#1}%
    \label{#2}%
}


% DEBUG variable
\newif\ifDEBUG

% Comment out to make DEBUG false and remove todos/notes
\DEBUGtrue

\ifDEBUG
    \newcommand{\todo}[1]{
        \noindent\red{ToDo: \textit{#1}}
        \\
    }

    \newcommand{\note}[1]{
        \noindent\textcolor{cyan}{{\textbf{Note} \textit{#1}}}
        \\
    }

    \newcommand{\inlinetodo}[1]{
        \red{\textbf{(#1)}}
    }
\else
    \newcommand{\todo}[1]{}
    \newcommand{\note}[1]{}
    \newcommand{\inlinetodo}[1]{}
\fi
%numeric
\usepackage[backend=biber,style=alphabetic]{biblatex}
\addbibresource{library.bib}

\begin{document}
  \makethesistitle
  %\affidavit{David Marlon Gengenbach}
  
\chapter*{Thesis Statement}
\vskip4ex
Pursuant to \S 23 paragraph 7 of APB TU Darmstadt, I herewith formally declare that I, David Marlon Gengenbach, have written the submitted thesis independently. I did not use any outside support except for the quoted literature and other sources mentioned in the paper. I clearly marked and separately listed all of the literature and all of the other sources which I employed when producing this academic work, either literally or in content. This thesis has not been handed in or published before in the same or similar form.

I am aware, that in case of an attempt at deception based on plagiarism (\S38 Abs. 2 APB), the thesis would be graded with 5,0 and counted as one failed examination attempt. The thesis may only be repeated once.
In the submitted thesis the written copies and the electronic version for archiving are identical in content.
\vskip8ex
\noindent Darmstadt, 18th of January 2018
\vskip6ex
\noindent\tudrule[0.6\textwidth]\\
{\normalsize(David Marlon Gengenbach)}
  
  \newpage
  
  \tableofcontents
  \newpage
  \setcounter{page}{1}
  \abstract{Text classification is an important natural language processing task. 
Here, finding appropriate representations for text content is crucial and directly influences the success in solving the task.
While state-of-the-art count-based representations like Bag-Of-Words perform well on numerous tasks, they often go hand-in-hand with structural information loss of their underlying text, eg. the semantic or syntax.
While one can augment these count-based representations to partially solve these issues, in this work, we instead explore another, more structured text representation, namely concept maps or concept graphs.
Automatically generated concept maps have successfully been employed to summarize text and therefor might capture important concepts and their relations.
This work explores whether and how one can harness this structural information in the context of text classification.
For this, we utilize different graph kernel-based approaches to extract features from the concept maps and examine their usefulness in conjunction with conventional text classification approaches.}
  \newpage
  \labelchapterinclude{Introduction}{sec:introduction}{chapters/introduction}
  \labelchapterinclude{Background}{sec:background}{chapters/background}
  \labelchapterinclude{Experimental Setup}{sec:evaluation}{chapters/evaluation}
  \labelchapterinclude{Results and Discussion}{sec:results}{chapters/results}
  \clearpage
  \labelchapterinclude{Conclusions}{sec:conclusions}{chapters/conclusions}
  %\listoftodos
  
  \newpage

  \printbibliography[heading=bibintoc]

%\vspace{3cm}
% \listoffigures
% \vspace{3cm}
% \listoftables
  
  \newpage
  \section*{Acknowledgements}
  I would like to acknowledge several people who helped me along the way.
  First, I'd like to thank Tobias Falke who gave me great support during the thesis with his answers and help structuring my thesis/work in general.
  Prof. Dr. Kristian Kersting, who provided me important hints and guided my thesis to a more scientific approach.
  My family, for their support and good words.
  Patrick Wieth and Marius Vieth, for awakening my interest in ML.
  Micha Ober, for the server he so graciously borrowed me for several months.
  Andreas Frankenberger, for providing a place for the borrowed server where it terrorized his employees with its non-bearable fan noises.
  And last but not least, the \textit{Neural Lourdes} friends for the interesting - and \textit{sometimes} even productive - discussions about ML.
  Also, thanks to all the contributers to open-source or free software who provide their work free of charge instead of creating pay-walls which arguably hinder progress in science.
  This work would not be possible in this form without open-source or free software.
  
  \vspace{0.5cm}
  \centering\textsf{\textbf{A thousand thanks to all of you!}}
\end{document}
